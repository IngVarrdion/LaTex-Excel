\documentclass{article}
\usepackage{graphicx} % Required for inserting images
\usepackage{amsmath} % For better math formatting

\title{Lab4}
\author{Ihor Pershehuba}
\date{05.12.2024}

\begin{document}

\maketitle

\section*{Task 1}

Album number: 72576. The first two and last two digits are 72 and 76.

\subsection*{a. Addition and Multiplication}

First, let's convert the numbers to binary format.

The decimal number 72:
\[
72 = 1001000_2
\]
The decimal number 76:
\[
76 = 1001100_2
\]

Now, we perform the arithmetic operations.

\textbf{Addition:} Add the binary numbers 1001000 and 1001100.

\[
1001000_2 + 1001100_2 = 10010100_2
\]

The result of the addition is \( 10011000_2 \), which is 148 in decimal.

\textbf{Multiplication:} Multiply the binary numbers 1001000 and 1001100.

\[
1001000_2 \times 1001100_2 = 1010101100000_2
\]

The result of the multiplication is \( \textbf{110010100000}_2 \), which is 5472 in decimal.

\subsection*{b. Subtracting the Larger Number from the Smaller Number}

Next, we subtract one number from the other in binary.

\textbf{Subtraction of 76 - 72:}

\[
1001100_2 - 1001000_2 = 0000100_2
\]

The result of the subtraction \( 76 - 72 = 4 \), which is \( 0000100_2 \) in binary.

\textbf{Subtraction of 72 - 76:} Since 72 is less than 76, the result will be negative. We use two's complement to represent negative numbers in binary.

\[
1001000_2 - 1001100_2 = 1111100_2
\]

This corresponds to \( -4 \) in decimal, since \( 1111100_2 \) is the two's complement representation of \( -4 \).

\subsection*{c. Division of the Last Two Digits}

Now, let's divide the last two digits: \(76_{10}\) and \(10_2\).
76 = 1001100

\[
\frac{1001100_2}{10_2} = 100110_2
\]

Since the album number ends with an even digit (6), we divide by \(10_{2}\). The result is approximately \(38_{10}\).

\section*{Task 2}

Now, let's represent the time 7:07:13 in binary format. We convert each number to binary.

The decimal number 7:
\[
7 = 111_2
\]
The decimal number 13:
\[
13 = 1101_2
\]

Now we build the binary clock. The time 7:25:12 corresponds to:

\[
\text{Hours: } 7 \quad (111_2), \quad \text{Minutes: } 7 \quad (111_2), \quad \text{Seconds: } 13 \quad (1101_2)
\]

We represent this on the binary clock, where 1 represents a black square and 0 represents a white square.

\[
\begin{array}{|c|c|c|c|c|c|}
\hline
0 & 1 & 0 & 1 & 0 & 0 \\
\hline
0 & 1 & 0 & 1 & 0 & 1 \\
\hline
0 & 1 & 0 & 1 & 1 & 1 \\
\hline
\end{array}
\]

\section*{Task 3}

The last two digits of the album number are 7 and 6. We change the sign of the last digit.

\[
\text{Last digit: } -7, \quad \text{Second-to-last digit: } 6
\]

Now we encode these numbers in the two's complement (U2) system using 4 bits.

The decimal number 7:
\[
7 = 0111_2
\]

The two's complement of -7:
\[
-7 = 1000_2
\]

The decimal number 6:
\[
6 = 0110_2
\]

Now we perform arithmetic operations with these encoded numbers in U2.

\textbf{Addition:}

\[
1000_2 + 0110_2 = 1101_2
\]

The result of the addition \( 7 + 6 = 13 \), which is \( 1101_2 \) in binary.

\textbf{Subtraction:}

\[
0110_2 - 0111_2 = 1111_2
\]

The result of the subtraction \( 6 - 7 = -1 \), which is represented as \( 1111_2 \) in two's complement.

\textbf{Multiplication:}

\[
0111_2 \times 0110_2 = 010110_2
\]

The result of the multiplication \( 7 \times 6 = 42 \), which is \( 010110_2 \) in binary.

\section*{Task 4}

Now we perform the division of numbers in two's complement (U2):

\[
\frac{12}{-4} = -3
\]

The decimal number 12:
\[
12 = 1100_2
\]

The two's complement of -4:
\[
-4 = 1011_2
\]

Now we perform the division of the two numbers in binary:

\[
\frac{1100_2}{1011_2} = 11111111_2
\]

The result of the division is \( -3 \), since \( 11111111_2 \) in two's complement represents \( -3 \).

\end{document}
