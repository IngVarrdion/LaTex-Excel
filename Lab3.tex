\documentclass{article}
\usepackage{amsmath}
\usepackage{amsfonts}

\title{Tasks on Number Systems and Codes}
\author{Pershehuba Ihor(w72576)}
\date{07.11.2024}

\begin{document}

\maketitle

\section*{Task 1}
I will define the range of numbers represented in the signed magnitude (SM) code for a 16-bit format. Based on the result, I will write the formula to calculate the range of numbers \( L \) in signed magnitude (SM) code in an \( n \)-bit format.

\textbf{Solution:}  
The range of numbers in signed magnitude code for a 16-bit format is:

\[
-2^{15} \leq L_{16} \leq 2^{15} - 1
\]

The general formula for the range of numbers in signed magnitude code in an \( n \)-bit format is:

\[
-2^{n-1} \leq L_n \leq 2^{n-1} - 1
\]

\section*{Task 2}
I will combine the first 8 digits of my PESEL number. I will assign a value of 0 to all even digits and a value of 1 to all odd digits and zeros. Then, I will calculate the decimal value of the resulting number, which has been encoded in an 8-bit signed magnitude code.

PESEL: 07221011130

- First 8 digits: `07221011`
- Apply encoding: Even digits = 0, Odd digits and 0 = 1:
\[
07221011 \rightarrow 10110101
\]
Now, I will convert the binary number `10110101` to decimal:
\[
10110101_2 = 181_{10}
\]

Thus, the decimal value is \( 181 \).

\section*{Task 3}
I will combine the 4th, 5th, and 6th digits of my PESEL number into a number and represent it in unsigned binary (U1) code. I will then convert the number into its opposite (negative) and write it again in U1 code.

PESEL (4th, 5th, and 6th digits):210

1. I will convert \( 210_{10} \) to binary :
   \[
   210_{10} = 11010010_2
   \]

2. I will convert the number to its negative (opposite):
   \[
   -210_{10} = 00101101_2 \quad
   \]

\section*{Task 4}
I will combine the first 8 digits of my PESEL number. In the resulting number, I will assign a value of 1 to all even digits and a value of 0 to all odd digits and zeros. Then, I will calculate the decimal value of the resulting number encoded in an 8-bit unsigned binary.

PESEL: 07221011130

- First 8 digits: `07221011`
- Apply encoding: Even digits = 1, Odd digits and 0 = 0:
\[
07221011 \rightarrow 11000000
\]
Now, I will convert the binary number `11000000` to decimal:
\[
11000000_2 = 192_{10}
\]

Thus, the decimal value is \( 192 \).

\section*{Task 5}
I will repeat the steps from Task 4 to obtain the number encoded in an 8-bit unsigned binary code and calculate the decimal value.

PESEL: 07221011130

- First 8 digits: `07221011`
- Apply encoding: Odd digits and zeros = 1, Even digits = 0:
\[
07221011 \rightarrow 11000000
\]
Now, I will convert the binary number `11000000` to decimal:
\[
11000000_2 = 192_{10}
\]

Thus, the decimal value is \( 192 \).

\section*{Task 6}
I will combine the last 3 digits of my PESEL number and represent them in unsigned binary code. I will convert the number to its opposite (negative) and write it again. I will also check the correctness of the result using the simplified method for converting decimal numbers.

PESEL (last 3 digits): 130

1. I will convert \( 130_{10} \):
   \[
   130_{10} = 10000010_2
   \]

2. I will convert the number to its opposite (negative):
   \[
   -130_{10} = 01111101_2 \quad 
   \]

\section*{Task 7}
I will convert the following numbers between number systems:

\begin{enumerate}
  \item \( 10111,01111_2 \) to decimal
  \item \( 1110,1100_2 \) to decimal
  \item \( 22,125_{10} \) to binary
  \item \( 75,875_{10} \) to binary
\end{enumerate}

\textbf{Solutions:}

a) \( 10111,01111_2 \rightarrow 7.46875_{10} \)

b) \( 1110,1100_2 \rightarrow 6.75_{10} \)

c) \( 22,125_{10} \rightarrow 10110.001_{2} \)

d) \( 75,875_{10} \rightarrow 1001011.111_{2} \)

\section*{Additional Task (Optional)}

\textbf{Task 1:}  
Assume that the 7th, 8th, and 9th digits of my PESEL number represent the integer part of a number, while the 10th and 11th digits represent the fractional part. Using fixed-point representation, I will convert the number into binary format with four decimal places of accuracy.

PESEL:07221011130

Integer part: `111`  
Fractional part: `30`

Number to convert: \( 111.30 \)

In binary:
\[
111.30_{10} \rightarrow 1101111.01001100110011001101_2
\]

\textbf{Task 2:}  
I will repeat the steps from Task 1 to obtain the decimal number. Using floating-point representation , I will convert it to binary with four decimal places of accuracy. I will check the correctness of the result.

PESEL:07221011130

Number to convert: \( 111.30 \)

In floating-point representation:
\[
111.30_{10} = 1.1130 \times 2^7 \quad 
\]

Thus, the binary representation is:
\[
1101111.01001100110011001101_2
\]

\end{document}
